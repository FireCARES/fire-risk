\documentclass[letterpaper,11pt]{article}
\usepackage{amsmath}
\begin{document}
\title{Tactical Correction Time (TCT) model}

\maketitle
\begin{abstract}
  Presently, many simple metrics exist that can be used to gauge the relative performance of fire departments. For example, the percentage of emergency incidents whose response time falls below a certain value, the number of unintentional fire deaths occurring in structures, or the percent of structure fires confined to room of origin are all used as metrics to judge a fire department�s performance. Aside from arrival time metrics, such metrics do not necessarily inform a fire department as to the effectiveness of their training and/or tactics because they do not supply a standard of comparison. For example, if 40\% of the structure fires a fire department responds to spread beyond the room of origin, it is unclear whether this is due to the fire department, or due to the types of fuel loads present in its community. There is a clear metric, time to arrival, that informs a fire department as to how it needs to train for response time, and what is missing regarding the speed at which they respond to an incident. Metrics as clear as time to arrival do not presently exist for determining quality of training and tactics regarding fire suppression. This paper posits a new metric, dubbed the tactical correction time (TCT), to aid in quantifying fire department tactical effectiveness. The concept is that there exists a TCT for all fire department structure fire responses that, in lieu of national standards governing response times and routed through a simple fire damage model, ``corrects'' the fire damage outcomes expected from national standards to the particular outcomes observed by a given fire department.  Here, ``fire damage outcomes'' refers to the structure fire spread category from the National Incident Reporting System (NFIRS), which rates the fire spread in a structure from ``confined to object of origin� through �spread beyond structure of origin.'' The ``departure'' of a fire department�s fire damage outcomes from those outcomes expected using national standards and a simple flame damage model provides an assessment metric that can identify where a given fire department�s training might be lacking, or where suppression tactics might be improved. Two models were built to explore this metric, one using optimization methods, and one using a Bayesian formulation.  These models were tested using fire data from the National Incident Reporting System (NFIRS), and Arlington County Fire Department (ACFD). Early results indicate an average TCT of 200 seconds for ACFD, meaning ACFD has, on average, 200 seconds worth of ``tactical correction,'' compared against a theoretically ``flawless'' fire department that possesses perfect information and tactics upon arrival at the scene.
\end{abstract}
\section{Introduction}
This document contains an exploration of a potential fire department assessment tool, deemed the Tactical Correction Time (TCT) model of fire department response. Two approaches have been used in examining this assessment tool. An optimization approach, and a bayesian hierarchical approach.
\section{TCT model}
The TCT model of a fire department response to a fire involves the following assumptions:
\begin{itemize}
  \item Fires are parametrized as follows: $\dot{Q}=At^\alpha$, where $\dot{Q}$ is the heat release rate of the fire (kW), $A$ is the initial size of the fire, $\alpha$ is the ``growth rate'' of the fire, and $t$ is the growth time of the fire.
  \item A fire department unit responding to a fire must complete some number of tasks that take up time before they can put water on the fire. For purposes of this model these tasks are  broken up as follows:
    \begin{itemize}
      \item Time to alarm - Time required before the fire is noticed, and some form of action is taken.
      \item Time to dispatch - Time required for dispatch operator to obtain enough information regarding the fire and location to issue a dispatch.
      \item Time to turnout - Time required for firefighter turnout.
      \item Time to arrival - Transit time required for engine between station and fire location.
      \item Time to suppress - Time required for firefighters on-scene to put water onto fire (includes size-up, hose connection, etc.)
    \end{itemize}
  \item When water is put onto a fire, the fire is assumed to have reached the peak size that it will ever reach. In other words, growth time for the fire is equivalent to the sum of the tasks the fire department must perform to suppress the fire. 
    \[t = t_{growth} = t_{alarm}+t_{dispatch}+t_{turnout}+t_{arrival}+t_{suppress} \]
  \item There exists some mapping between the size of a fire and the extent of its spread. For example, one may assume that a fire that grows beyond 2 MW will, guaranteed, spread beyond its room of origin.
\end{itemize}

In addition to these assumptions, data are available for a given fire department jurisdiction indicating the extent of fire spread in said jurisdiction's structure fire responses. A single fire response is denoted $X$, and is described as follows:
\[X = \left\{
      \begin{array}{ll}
	0 & \text{Fire contained to room of origin} \\
	1 & \text{Fire contained to building of origin} \\
	2 & \text{Fire spreads beyond building of origin}
      \end{array}
      \right.
\]
Thus, if $X=1$, that means a given fire was contained to the building of origin.

\section{Models}
This section contains the mathematical assumptions and derivations for the two models explored. The glossary of terms used for both models is contained in Table~\ref{tab:terminology}
\begin{table}
  \centering
  \begin{tabular}{lp{4cm}}
    $X$ & extent of fire spread \\
    $x_i$ & individual observation of extent of fire spread \\
    $t_{al}$ & alarm time \\
    $t_{di}$ & dispatch time \\
    $t_{to}$ & turnout time \\
    $t_{ar}$ & arrival time \\
    $t_{su}$ & suppression time \\
    $t_{cor}$ & correction time \\
    $A$ & Initial fire size \\
    $t_{gro}$ & Fire growth time \\
    $\alpha$ & Fire ``growth rate'' \\
    $\dot{Q}$ & Fire size (kW) \\
  \end{tabular}
  \caption{Glossary of terms used in model}
  \label{tab:terminology}
\end{table}

\section{Optimization approach}

The optimization method seeks to minimize the difference between a random sample from the ideal model and the true population of fires for a given fire department jurisdiction by using the correction time. This is accomplished by first calculating the total number of fires a jurisdiction has seen and randomly sampling this many fires from the growth model. A single value of $t_{cor}$ is then found that minimizes the sum of squared errors between the number of fires in each of the sampled bins and the observed number of fires in each bin. 

The steps are as follows:
\begin{itemize}
  \item Sample $n$ times from $A$, $\alpha$, and alarm, dispatch, turnout, arrival, and suppression times
  \item Use a golden section search for $t_{cor}$ with the objective function 
  \begin{equation}    
    min{\sum (N_{i,obs}-N_{i,pre})^2}
  \end{equation}
  \item Repeat this process to construct a population of $t_{cor}$ for the given jurisdiction
\end{itemize}

\section{Bayesian approach}

\subsection{Final expression}

\[P(t_{cor} | t_{al},t_{di},t_{to},t_{ar},t_{su},\alpha,A,x_i) \sim
  \begin{array}{lr}
    Unif(-840,689165) & : x_i = 0 \\
    Unif(-634,3447785) & : x_i = 1 \\
    Unif(-378.74,\infty) & : x_i = 2 \\
  \end{array}
\]
\end{document}
